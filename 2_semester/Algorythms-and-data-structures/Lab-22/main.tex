\documentclass[12pt]{report}
\usepackage[utf8]{inputenc}
\usepackage[russian]{babel}
\usepackage{ upgreek }
\usepackage{ amsmath }
\usepackage{ stmaryrd }
\usepackage{ tipa }
\usepackage{ gensymb }
\usepackage{ wasysym }
\setcounter{page}{30}
\usepackage{ dsfont }
\usepackage{ textcomp }

\begin{document}

\section*{\large{1.7 Обработка информации}}


\footnotesize{Множество сообщений $N$ представляет интерес только тогда, когда ему соответствует (по крайней мере одно) множество сведений $I$ и определено соответствующее правило интерпретации $\varphi:N \to I$ (см.п. 1.2). Так как множеству сообщений $N'$ тоже соответствует некоторое множество сведений $I'$ (и правило интерпретации $\varphi'$), то любоу правило обработки сообщений $\nu:N \to N'$ (см. п. 1.6) приводит к следующей диаграмме:
\newline
\begin{math*}


    \qquad \qquad \quad N \overset{\varphi}{\to} I

     
    \qquad \qquad {\nu}\shortdownarrow \qquad \Downarrow{\sigma}

     
   \qquad \quad \qquad N' \overset{\varphi'}{\to} I' \qquad \qquad \qquad \qquad \qquad \qquad \qquad \qquad \qquad \qquad \qquad \quad \eqno(*)
   \newline
   
\end{math*}
Эта диаграмма определяет соответствие между множествами $I$ и $I'$. Так как согласно диаграмме (*) каждому сообщению $n \in N$ соответствует пара сведений $i = \varphi(n) \in I$ и $i' = \varphi(\nu(n)) \in I'$, построенное соответствие между $I$ и $I'$ (обозначим его через $\sigma$), вообще говоря, не является отображением. В самом деле, если правило интерпретации $\varphi$ не является однозначным (инъективным, когда разные переходят в разные), т.е. если существуют два разлинчых сообщения $n_1,n_2 \in N, n_1 \textdoublebarslash n_2$, передающих одинаковую информацию $i = \varphi(n1) = \varphi(n2)$, то может оказаться, что $\varphi'(\nu(n1)) \textdoublebarslash \varphi'(\nu(n2))$ и, следовательно, одной информации $i \in I$ будут соответствовать (по крайней мере) две различных информации $i'_1 = \varphi'(\nu(\varphi^-1(i))$.


Во всех случаях, когда соответствие $\sigma$ является отображением, правило обработки сообщений $\nu$ называется \textbf{сохраняющим информацию}. Если правило обработки сообщений $\nu$ сохраняет информацию, то диаграмма
\newline
\begin{math*}


    \qquad \qquad \quad N \overset{\varphi}{\to} I

     
    \qquad \qquad {\nu}\shortdownarrow \qquad \shortdownarrow{\sigma}

     
   \qquad \quad \qquad N' \overset{\varphi'}{\to} I' \qquad \qquad \qquad \qquad \qquad \qquad \qquad \qquad \qquad \qquad \qquad \quad \eqno(**)
   \newline
   
\end{math*}
коммутативна: $\nu \ocircle \varphi' = \varphi \ocircle \sigma$. Отображение $\sigma$ называется в этом случае \textbf{правилом обработки информации}.


Обычно обработку информации сводят к обработке сообщений, т. е., исходя из требуемого правила обработки информации $\sigma$, пытаются определить отображения $\nu$, $\varphi$ и $\varphi'$ таким образом, чтобы диаграмма (**) была коммутативной.


Если $\sigma$ - обратимое (взаимно однозначное) отображение, т. е. если информация при обработке по правилу $\sigma$ не теряется, то соответствующую обратку сообщений $\nu$ называют \textbf{перешифровкой}.


Пусть $\nu$ - обратимая перешифровка. Тогда по сообщению $n' = \nu(n)$ можно восстановить не только исходную информацию, но и само исходное сообщение $n$. Иными словами, в этом случае $n'$ \textit{кодирует} $n$ (см. п. 1.4). Обратимая перешифровка $\nu$ называется \textbf{перекодировкой}.


Пусть перешифровка $\nu$ не является обратимой, т. е. пусть несколько сообщений из $N$ копируются одним и тем же сообщением из $N'$. Но так как при перешифровке информация не теряется, это означает, что исходное множество сообщений $N$ является избыточным: некотоые сообщения из $N$ содержат одну и ту же информацию (дублируют друг друга). В $N'$ таких дублирующих сообщений меньше, чем в $N$, так как при обработке по правилу $\nu$ некоторые из дублирующих друг друга сообщений <<сливаются>> в одно сообщение. Перешифровка $\nu$, которая не является обратимой, называется \textbf{сжимающей}. Cжатию подвергается множество сообщений. То есть в результате необратимой перешифровки сообщений их количество уменьшается, а информация может либо сохраняться, либо теряться.


\textbf{Пример 1.7.1} Пусть сообщения $(a,b)$, составленные из пар целых чисел (например, в десятичной позиционной записи), передают информацию <<рациональное число $r$, представленное дробью $\frac{a}{b}$>>. Тогда $N = \mathds{Z} \times \mathds{N}$ (где $\mathds{Z}$ - множество целых чисел, \mathds{N} - множество, натуральных чисел), $I = \mathds{Q}$ (\mathds{Q} - множество рациональных чисел). Отображение $\varphi:N \to I$ не является обратимым, так как при любом целом $n$ парам ($a,b$) и ($na,nb$) соответствует одно и то же рациональное число $r$. Пусть $N'$ - множество пар ($p, q$) взаимно простых целых чисел и пусть $\nu:N \to N'$ переводит все ($np,nq$) в ($p,q$). Тогда $\nu$ - сжимающее отображение, а $\varphi':N' \to I$ - обратимое отображение (мы считаем $I' = I$). Такое отображение $\nu$ называется \textbf{вполне сжимающей перешифровкой}, поскольку после обработки сообщений соответствие между сообщениями и информацией биективно. Здесь информация не теряется.


Если $\sigma$ - необратимое отображение, т. е. если разные сведения из $I$ отображаются в одну и ту же информацию $i' \in I'$. В этом случае производится \textit{выбор} из данного множества сведений.


Таким образом, <<обработка информации>> - это,как правило, сокращение количества информации. Во всяком случае, верно утверждение: обработка информации никогда \textbf{не добавляет} информацию, она состоит в том, что \textbf{извлекает} интересную информацию из той, которая содержится в сообщении.
}


\section*{\large{Лекция 4}}


\section*{\large{1.8 Автоматизация обработки информации}}


Вернёмся к рассмотрению диаграммы (**).Если заменить на ней отображение $\varphi$ обратным отображением $\psi=\varphi^-1$, получим новую диаграмму:
\newline
\begin{math*}


    \qquad \qquad \quad N \overset{\psi}{\to} I

     
    \qquad \qquad {\nu}\shortdownarrow \qquad \shortdownarrow{\sigma}

     
   \qquad \quad \qquad N' \overset{\varphi'}{\to} I' \qquad \qquad \qquad \qquad \qquad \qquad \qquad \qquad \qquad \qquad \qquad \quad
   \newline

\end{math*}
Автоматизация обработки информации заключается в выполнении $\sigma$ или $\varphi^-1 \ocircle \nu \ocircle \varphi'$ при помощи физических устройств. Однако в программировании изучаются методы автоматического выполнениия только отображения $\nu$, т. е. обработки сообщений. Программно-аппаратная реализация отображений $\psi,\varphi'$ изучается в другом разделе информатики, который называется <<Искусственный интеллект>> - и потому выходят за рамки нашего курса.


\end{document}
